%%%%%%%%% EXPERIMENTS
\section{Experiments}
\paragraph{Datasets.}
For all datasets, given images/videos and natural language text as input, the system is expected to predict a binary class label.
NLVR$^2$~\citep{suhr2019corpus} contains ${\sim}86K, 7K, 7K$ samples for training, development, and testing respectively.
Each sample in NLVR$^2$ consists of a pair of images (from search engines) and a sentence (crowd-sourced).
VIOLIN~\citep{liu2020violin} contains video clips from popular TV shows and movies along with subtitles and crowd-sourced statements.
VIOLIN contains $76K$, $9.5K$, $9.5K$ samples for training, validation and testing.
VQA Yes/No consists of image-question-answer triplets from VQA-v2 dataset~\citep{goyal2017making}.
While VQA-v2 consists of multiple question and answer types, we focus on the subset of questions with binary \textit{yes/no} answers (${\sim}38\%$ of VQA-v2).
    
\paragraph{Evaluation Metrics.}
We use two evaluation metrics:
(1) \textbf{Clean Accuracy}: accuracy on the i.i.d.\ benchmark test set, and
(2) \textbf{SISP Accuracy}: average performance on SISP transformations of the test set.
Since SISP transformations are automated and can be noisy (Sec~\ref{sec:sisp_fidelity_bias}), evaluation on the SISP test set can be considered a proxy for robustness.

\footnotetext[3]{
Notation: \textbf{bold}: higher than SOTA;
shaded: higher than respective backbone;
\uline{underlined}: best SI/SP accuracies.
}

\subsection{Results}
We compare SDRO with backbone models that use standard training data \textit{(BASE)} and data-augmentation \textit{($+$data-aug)}.
We train SDRO and backbones with the same hyperparameters.
We apply test-time ensembling to the best SDRO model.
\paragraph{NLVR$^2$:}
\input{sdro/tables/results_NLVR2}
We use Transformer-based models LXMERT~\citep{tan2019lxmert}, UNITER~\citep{chen2020uniter}, and VILLA~\citep{gan2020large} as backbones for SDRO. 
VILLA (the current state-of-the-art for NLVR$^2$) uses standard adversarial training.
The percentage of SISP-transformed samples is fixed at $T{=}20\%$.
Table~\ref{tab:results_nlvr2} shows results on the NLVR$^2$ test set, with consistent model-agnostic improvements in clean accuracy over each baseline model and improved robustness on average.
Both variants of SDRO improve over VILLA$_{BASE}$ by $0.84\%$ and $1.02\%$, respectively.
Test-time ensembling using Equation~\ref{sdro:eq:ensembling} leads to further gains, resulting in a new state-of-the-art accuracy of $82.22\%$, an improvement of $3.83\%$ over VILLA$_{BASE}$.
GW-SDRO results in the highest SI accuracy when used with each backbone model.



\paragraph{VIOLIN:}
\begin{table}
    \centering
    % \footnotesize
    % \resizebox{0.9\linewidth}{!}{
    \begin{tabular}{@{}lcccc@{}}
        \toprule
        \multirow{2}{*}{\textbf{Model}} & \multirow{2}{*}{\textbf{Clean Acc.}} & \multicolumn{3}{c}{\textbf{SISP Acc.}} \\
        \cmidrule{3-5}
         &  & SP & SI & Avg. \\
        \midrule
        VIOLIN$_{BASE}$              & 68.07 & 57.17 & 57.20 & 57.18 \\           
        \qquad + data-aug            & 61.58 & \textul{67.64} & 67.70 & 67.67 \\
        \qquad + SW-SDRO             & 62.81 & 62.84 & 62.68 & 62.76 \\
        \qquad + GW-SDRO             & 63.71 & 64.58 & 63.16 & 63.87 \\
        \qquad \qquad + Ensemble     & 66.56 & --''-- & --''-- & --''-- \\
        \midrule
        HERO$_{BASE}$               & 68.55 & 65.59 & 32.00 & 48.80 \\
        \qquad + data-aug            & 65.21 & 59.20 & 81.81 & \textul{70.51} \\
        \qquad + SW-SDRO    & \cellcolor{VeryLightGray}{\textbf{68.83}} & 58.97 & 77.83 & 68.41 \\
        \qquad + GW-SDRO             & 68.19 & 56.20 & \textul{82.92} & 69.57 \\
        \qquad \qquad + Ensemble     & \cellcolor{VeryLightGray}{\textbf{69.90}} & --''-- & --''-- & --''-- \\
        \bottomrule
    \end{tabular}
    % }
    \caption[Results on VIOLIN test set.]{Results on VIOLIN.\footnotemark[\value{footnote}]}
    \label{tab:results_violin}
\end{table}
We consider VIOLIN$_{BASE}$~\citep{liu2020violin} and HERO~\citep{li2020hero}, the current state of the art, as baselines.
VIOLIN$_{BASE}$ separately computes visual features using Faster-RCNN~\citep{ren2015faster} and textual features using BERT~\citep{devlin2019bert}, and fuses them to be used as input to a classifier model.
On the other-hand, HERO is a large-scale transformer-based pre-trained model which uses various V\&L pre-training tasks to compute cross-modal features.
We set $T{=}40\%$.
The results can be seen in Table~\ref{tab:results_violin}.
SW-SDRO model with the HERO backbone improves the state-of-the-art to $68.83\%$, and test-time ensembling further improves it to $69.90\%$. 
Interestingly, similar improvements in clean accuracy are not observed for VIOLIN$_{BASE}$, potentially because it does not use cross-modal pre-trained features.

\paragraph{VQA Yes/No:}
\begin{table}
    \centering
    % \resizebox{0.9\linewidth}{!}{
    \begin{tabular}{@{}lcccc@{}}
        \toprule
        \multirow{2}{*}{\textbf{Model}} & \multirow{2}{*}{\textbf{Clean Acc.}} & \multicolumn{3}{c}{\textbf{SISP Acc.}} \\
        \cmidrule{3-5}
         &  & SP & SI & Avg. \\
        \midrule
        % % LXMERT (all data)           & 86.65 & 75.41 & 34.09 & 54.75\\
        % LXMERT\footnote{LXMERT and UNITER follow a different train-val split}                      & 82.71 & 75.53 & 33.32 & 53.92\\
        % LXMERT\footnotemark[1] + LOL                & 84.75 & 73.77 & 42.97 & 58.37 \\
        % LXMERT-VILLA              & {83.85} & 76.70 & 34.68 & 55.59 \\
        % \midrule
        UNITER$_{BASE}$             & 83.49 & 72.04 & 38.90 & 55.47 \\
        \qquad + data-aug           & 82.53 & 77.03 & 93.70 & 85.36 \\
        \qquad + SW-SDRO    & \cellcolor{VeryLightGray}{83.92} & 75.82 & 88.92 & 81.48 \\ 
        \qquad + GW-SDRO     & \cellcolor{VeryLightGray}{84.05} & 76.95 & 93.41 & 85.18 \\
        \qquad \qquad + Ensemble     & \cellcolor{VeryLightGray}{84.22} & --''-- & --''-- & --''-- \\
        \midrule
        VILLA$_{BASE}$              & 84.82 & 74.15 & 37.40 & 55.77 \\
        \qquad + data-aug            & 83.54 & \textul{78.33} & \textul{94.55} & \textul{86.45} \\
        \qquad + SW-SDRO    & 84.54 & 74.02 & 88.32 & 81.17 \\
        \qquad + GW-SDRO     & \textbf{\cellcolor{VeryLightGray}{{85.12}}} & 77.92 & 93.42 & 85.67 \\
        \qquad \qquad + Ensemble     & \cellcolor{VeryLightGray}{\textbf{85.37}} & --''-- & --''-- & --''-- \\
        \bottomrule
    \end{tabular}
    % }
    \caption[Results on the VQA yes/no subset.
    Not to be compared with VQA-v2 leaderboard since we use a smaller training set.]{Results on the VQA yes/no subset.\footnotemark[\value{footnote}] Not to be compared with VQA-v2 leaderboard since we use a smaller training set of \textit{yes/no} questions.
    % Note that these results should not be compared with VQA-v2 leaderboard since we use a smaller training set.
    }
    \label{tab:results_vqayesno}
\end{table}
We use UNITER and VILLA as the backbone models, with $T{=}20\%$.
The motivation behind VQA experiments is to show that SISP transforms and SDRO can be extended to other V\&L tasks.
Table~\ref{tab:results_vqayesno} shows that GW-SDRO is the best performing model in terms of clean accuracy, and is further improved by test-time ensembling.


