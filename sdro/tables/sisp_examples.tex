\begin{table}
    \centering
    % \small
    \resizebox{\linewidth}{!}{
    \begin{tabular}{@{}clp{0.425\linewidth}p{0.425\linewidth}@{}}
        \toprule
         & \textbf{Category}    & \textbf{Original} & \textbf{Transformed} \\
        \toprule 
        \multirow{6}{*}{\rotatebox{90}{SI}}
         & Noun-Antonym
            & The two women are driving on the street with the convertible top down.
            & The two \textit{\textbf{men}} are driving on the street with the convertible top down.\\
         & Verb-Antonym
            & There are children standing by the door.
            & There are children \textbf{\textbf{sitting}} by the door.\\
         & Comparative-Antonym
            & There are more monitors in the image on the right than on the left.
            & There are \textit{\textbf{few}} monitors in the image on the right than on the left. \\
         & Number-Substitution
            & There are three bowls of dough with only one spatula.
            & There are \textit{\textbf{eleven}} bowls of dough with only one spatula. \\
         & Pronoun-Substitution 
            & In one of the images, a woman is taking a selfie. 
            & In one of the images, \textit{\textbf{he}} is taking a selfie.\\
         & Subject-Object Swap
            & The two women are driving on the street with the convertible top down. 
            & The two \textit{\textbf{top}} are driving on the street with the convertible \textit{\textbf{women}} down.\\
         & Negation
            & The closet doors on the right are mirrored.
            & The closet doors on the right are \textit{\textbf{not}} mirrored\\
        \midrule
        \multirow{6}{*}{\rotatebox{90}{\footnotesize SP}}
         & Noun-Synonym         
            & The right image shows three bottles of beer lined up.
            & The right \textit{\textbf{picture}} shows three bottles of beer lined up.\\
         & Verb-Synonym         
            & Someone is using a kitchen utensil
            & Someone is \textit{\textbf{utilizing}} a kitchen utensil. \\
         & Comparative-Synonym  
            & The bottle on the right is larger than the bottle on the left.
            & The bottle on the right is \textit{\textbf{bigger}} than the bottle on the left.\\
         & Number-Substitution  
            & The two white swans are swimming in the canal gracefully. 
            & The \textit{\textbf{less than seven}} white swans are swimming in the canal gracefully.\\ 
         & Pronoun-Substitution 
            & In one of the images, a woman is taking a selfie. 
            & In one of the images, \textit{\textbf{she}} is taking a selfie.\\
         & Paraphrasing         
            & A man in a green shirt came on the porch and knocked on the door.
            & A man in a green shirt came \textit{\textbf{up to}} the porch and knocked on the door. \\
        \bottomrule 
    \end{tabular}
    }
    \caption{Illustrative examples for the effect of each SISP transformation on input sentences.}
    \label{tab:sisp_examples}
\end{table}