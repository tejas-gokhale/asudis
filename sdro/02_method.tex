%%%%%%%%% SDRO
\section{Method} 
\begin{table}
    \centering
    % \small
    \resizebox{\linewidth}{!}{
    \begin{tabular}{@{}clp{0.425\linewidth}p{0.425\linewidth}@{}}
        \toprule
         & \textbf{Category}    & \textbf{Original} & \textbf{Transformed} \\
        \toprule 
        \multirow{6}{*}{\rotatebox{90}{SI}}
         & Noun-Antonym
            & The two women are driving on the street with the convertible top down.
            & The two \textit{\textbf{men}} are driving on the street with the convertible top down.\\
         & Verb-Antonym
            & There are children standing by the door.
            & There are children \textbf{\textbf{sitting}} by the door.\\
         & Comparative-Antonym
            & There are more monitors in the image on the right than on the left.
            & There are \textit{\textbf{few}} monitors in the image on the right than on the left. \\
         & Number-Substitution
            & There are three bowls of dough with only one spatula.
            & There are \textit{\textbf{eleven}} bowls of dough with only one spatula. \\
         & Pronoun-Substitution 
            & In one of the images, a woman is taking a selfie. 
            & In one of the images, \textit{\textbf{he}} is taking a selfie.\\
         & Subject-Object Swap
            & The two women are driving on the street with the convertible top down. 
            & The two \textit{\textbf{top}} are driving on the street with the convertible \textit{\textbf{women}} down.\\
         & Negation
            & The closet doors on the right are mirrored.
            & The closet doors on the right are \textit{\textbf{not}} mirrored\\
        \midrule
        \multirow{6}{*}{\rotatebox{90}{\footnotesize SP}}
         & Noun-Synonym         
            & The right image shows three bottles of beer lined up.
            & The right \textit{\textbf{picture}} shows three bottles of beer lined up.\\
         & Verb-Synonym         
            & Someone is using a kitchen utensil
            & Someone is \textit{\textbf{utilizing}} a kitchen utensil. \\
         & Comparative-Synonym  
            & The bottle on the right is larger than the bottle on the left.
            & The bottle on the right is \textit{\textbf{bigger}} than the bottle on the left.\\
         & Number-Substitution  
            & The two white swans are swimming in the canal gracefully. 
            & The \textit{\textbf{less than seven}} white swans are swimming in the canal gracefully.\\ 
         & Pronoun-Substitution 
            & In one of the images, a woman is taking a selfie. 
            & In one of the images, \textit{\textbf{she}} is taking a selfie.\\
         & Paraphrasing         
            & A man in a green shirt came on the porch and knocked on the door.
            & A man in a green shirt came \textit{\textbf{up to}} the porch and knocked on the door. \\
        \bottomrule 
    \end{tabular}
    }
    \caption{Illustrative examples for the effect of each SISP transformation on input sentences.}
    \label{tab:sisp_examples}
\end{table}

\subsection{Preliminaries}
Consider a training distribution $P_{tr}$ consisting of inputs $\x$ and labels $\y$.
For VLI, input $\x$ is multi-modal (visuals and text), with labels $\y \in \{\texttt{True}, \texttt{False}\}$.
Under the empirical risk minimization (ERM), the following risk is minimized:
\begin{equation}
    \mathcal{R}_{ERM} = \E_{(\x, \y) \sim P_{tr}}  ~\ell(f(\x; \theta), \y).
    \label{sdro:eq:erm}
\end{equation}
ERM provides generalization guarantees~\citep{vapnik1991principles} for i.i.d.\ test samples, but not for out-of-distribution or adversarial examples.

\paragraph{Distributed Robust Optimization (DRO)}~\citep{pmlr-v80-hu18a,sagawa2020distributionally} searches for loss-maximizing perturbations of the input within an $\epsilon$-divergence ball around $P_{tr}$ and minimize the risk over such perturbed distributions.
\begin{equation}
    \mathcal{R}_{DRO}=\underset{P:D(P, P_{tr})<\epsilon}{sup}\E_{(\x, \y) \sim P} \ell(f(\x; \theta), \y).
    \label{sdro:eq:std_adv}
\end{equation}
The solution to Equation~\ref{sdro:eq:std_adv} guarantees robustness inside such $\epsilon$-bounded distributions $P$.
The inner maximization is typically solved using gradient-based methods~\citep{madry2018towards} over additive perturbations $\delta$ such that $\x+\delta$ fools the classifier.


\subsection{SDRO}
\label{sec:sdro}
For sentence inputs, additive perturbations are intangible and may result in ambiguity.
An alternative approach is to consider \textit{groups} or perturbations sets $\mathcal{G}$ representing certain subpopulations or semantic categories within the data distribution.
For text inputs, we consider perturbation sets that can be created using semantic sentence transformations such as those shown in Table~\ref{tab:sisp_examples}, as the \textit{groups} (or equivalently, \textit{transformations}).
These transformations $g(x, y) = (\x_g, \y_g)$ are of two types:
semantics-preserving (SP) if $\y_g{=}\y$, or semantics-inverting (SI) if $\y_g \neq \y$.
The ability of generating adversarial samples with inverted meanings is a key distinction between adversarial training (AT) and SDRO.
While AT is restricted to SP perturbations inside an $\epsilon$ norm-ball, SDRO can impart larger linguistic perturbations (both SI and SP) beyond the norm-ball, by minimizing the worst-case expected risk over these groups:
\begin{equation}
    \mathcal{R}_{SDRO} = \underset{g\in\mathcal{G}}{sup}\E_{(\x, \y) \sim g} ~\ell(f(\x; \theta), \y).
    \label{sdro:eq:sdro}
\end{equation}


\paragraph{Implementation.}
As a first step of SDRO, we randomly sample a subset $\mathcal{C}$ of the training dataset $\mathcal{D}$ s.t. $|\mathcal{C}|/|\mathcal{D}|=T$. 
We find adversarial samples after every epoch and create an augmented dataset $\mathcal{D}_{aug}$ which contains $(1-T)|\mathcal{D}|$ original samples and $T|\mathcal{D}|$ adversarial samples, thus retaining the size of the training dataset.
We define $\ell_g$ as the classification loss for a transformed sample $(\x_g, \y_g)$:
\begin{equation}
    \ell_g(\x, \y) \triangleq \ell(f(\x_g), \y_g),~\forall g{\in}\mathcal{G}.
    \label{sdro:eq:ell_G}
\end{equation}
We design two variants of SDRO: Sample-Wise (SW) and Group-Wise (GW).
\paragraph{Sample-Wise SDRO:}
    In this greedy version of SDRO, for every input $\x$, a transformation that maximally fools the classifier: $g^*{=}\argmax_{g\in\mathcal{G}} \ell_g(\x, \y)$, is added to the set of adversarial examples $\mathcal{D}_{adv}$. The model is then fine-tuned on the augmented dataset.
    \begin{align}
        \mathcal{D}_{adv} &= \{ g^*(\x, \y) \colon (\x, \y){\in}~\mathcal{C}\}, \\
        \mathcal{D}_{aug} &= \mathcal{D}_{1:(1-T)|\mathcal{D}|} \cup \mathcal{D}_{adv}
        \label{sdro:eq:d_aug}
    \end{align}
    However, this greedy approach is susceptible to the model's biases towards certain transformations.  
    For instance, if negation and verb-antonym are universally hard for most sentences, i.e., result in the maximum classifier loss amongst all  transformations $g$, then $\mathcal{D}_{adv}$ will be dominated by these groups, resulting in an unbalanced training set.
    
\paragraph{Group-Wise SDRO}
is devised to mitigate against the model becoming biased towards the ``hardest'' transformations.
    Using Equation~\ref{sdro:eq:ell_G}, we calculate the transformation losses for each transformation of each sample in a training batch, yielding a set of classifier losses per ``group'' $g$:
    \begin{equation}
        L_g \colon \mathcal{C} \rightarrow \R;
        ~L_g = \{ \ell_g(\x, \y) \colon (\x, \y) \in \mathcal{C}\}.
    \end{equation}
    We obtain the top-k losses per group $g$ as:
    \begin{equation}
        L_G^k = \argmax_{\Lambda \subset L_G, |\Lambda| = k} \sum_{\lambda\in\Lambda} \lambda,
        ~~\text{where~}k = \floor*{\frac{|\mathcal{C}|}{|\mathcal{G}|}}.
        \label{sdro:eq:pergroup_losses}
    \end{equation}
    Then 
    $\mathcal{D}_{adv}$ is compiled as the union of per-group adversaries using Equation~\ref{sdro:eq:pergroup_losses}, and augmented to the training dataset using Equation~\ref{sdro:eq:d_aug}.
    
\paragraph{Test-Time Ensembling of Predictions.}
Semantic transformations $g$ allow us to obtain multiple ``views'' $\x_g = g(\x)$ of the input, and the corresponding predictions $\hat{\y}_g = f(\x_g)$.
We ensemble these predictions and the original prediction $\hat{\y}=f(\x)$ with a simple weighted-average.
Note that $\mathcal{G}$ contains both SP and SI transformations, $\mathcal{G}_{SP}$ and $\mathcal{G}_{SI}$. 
Since the expected label for $\mathcal{G}_{SI}$ is flipped, during ensembling we use the flipped probabilities $1{-}f(\x_g)$.
The ensembled prediction is:
\begin{equation}
\hat{\y}_{e} = \alpha f(\x) +
\frac{1{-}\alpha}{2}\sum_{\mathclap{g\in\mathcal{G}_{SP}}}{\frac{f(\x_g)}{|\mathcal{G}_{SP}|}} + \frac{1{-}\alpha}{2}\sum_{\mathclap{g\in\mathcal{G}_{SI}}}\frac{1{-}f(\x_g)}{|\mathcal{G}_{SI}|}.
\label{sdro:eq:ensembling}
\end{equation}
This ensembling is in principle similar to~\citet{chai2021ensembling} who train a generative model $g$ to output different views of an image, and tune $\alpha$ over a validation set.
In our work, $g$ are semantic sentence transformations, and a simple intuitive choice of $\alpha{=}0.5$ gives equal weight to the original sample and the SISP versions.
We find that:
\begin{enumerate}[nosep,noitemsep,leftmargin=*]
\item training models with SDRO using SISP transformations improves results on VLI tasks, and
\item ensembling predictions of SDRO at test-time using Equation~\ref{sdro:eq:ensembling} further improves results.
\end{enumerate}