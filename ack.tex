\acknowledgementpage{\\

{
\begin{FlushRight}
    ``Knowledge is in the end based on acknowledgement.''\\
    - Ludwig Wittgenstein\\
    \medskip
    \hrule 
    \bigskip
\end{FlushRight}
}

\begin{justify}
I would like to express my deepest gratitude to my advisors Yezhou Yang and Chitta Baral, who have been an unconditional source of encouragement, and a crucial support system before, during, and after the pandemic.  Yezhou and Chitta offered a rather delicious blend of advise, offering up combo-style meals of big picture messaging + practical and grounded methods,  vision + language (literally and metaphorically).
% \textit{karma-yog} (the path of action) + \textit{jnana-yog} (the path of knowledge).  
While most of my Ph.D.\ journey was clouded by the pandemic, Yezhou and Chitta offered me their unflinching support and encouragement and helped me navigate myself out of local minima.  They created a healthy work environment that allowed me to explore, stumble, and to learn to pick myself up.  It has been an absolute privilege to learn, collaborate, dream, and disagree with Yezhou and Chitta.


I am also grateful for the mentorship of Rushil Anirudh at Lawrence Livermore National Laboratory.
The terrific trio of Rushil, Jay Thiagarajan, and Bhavya Kailkhura introduced me to many ideas in robustness and generalization, making my two summers ``at'' LLNL a game-changer for my Ph.D.\ thesis. 


Microsoft


I am indebted to the knowledge I received from exemplary \textit{gurus} -- creators and sustainers of knowledge and destroyers of ignorance.
I have been fortunate to have learned Machine Learning from Nina Balcan and Gautam Dasarathy, Deep Learning from Ruslan Salakhudtinov, Convex Optimization from Aarti Singh, Computer Vision from Deva Ramanan and Suren Jayasuriya, Speech and Natural Language Processing from Alan Black, and a re-education in signal processing from Aswin Sankaranarayanan. 
I would like to thank Aswin for advising me at Carnegie Mellon University and and Kuldeep Kulkarni who mentored my first research project of substance.
Aswin introduced me to the art of research and was the catalyst in my decision to pursue a Ph.D.
Kuldeep's knack of turning whiteboard sketches (or chess-board enactments) into concrete research ideas was inspirational.


And then are the unsung heroes of ASU -- Pamela Dunn and Monica Dugan in the SCAI Business Office, Jaya Krishnamurthy and Arzuhan Kavak in SCAI Advising, Theresa Chai in SCAI Finance, Lincoln Slade and Brint MacMillan in SCAI IT, Katrina Roalson and Pamela Felix in ASU Graduate College, Sydney Burt in GPSA, and ASU ISSC for their priceless support in navigating logistics, computing and infrastructure, administrative issues, conference travel, fellowships and payroll.


I also owe much to my wonderful collaborators at ASU -- Pratyay Banerjee, Zhiyuan Fang, Man Luo, Swaroop Mishra, Neeraj Varshney, Joshua Feinglass, Ethan Wisdom, Sheng Cheng and my talented mentees Abhishek Chaudhary, Huiliang Shao, Maitreya Patel, and Agneet Chatterjee who kept me on my toes.
I would also like to thank my friends Lu Cheng, Kowshik Thopalli, John Janiczek, Mohammad Farhadi, Changhoon Kim, Varun Jammula, Rudra Saha, Sam Rawal, Aurgho Bhattacharjee, Ishan Shrivastava, Kuntal Pal, Mihir Parmar, Raha Moraffah, Kaize Ding, and all members of the APG, CogInt, and Summer Vision Reading Groups.
I am also thankful for the \textit{TFG++}, whom I consider to be family.
I would like to mention my appreciation for my bookshelf which has grown to insane proportions;
% but has been often (and unfortunately) ignored and replaced by Netflix, and YouTube. It 
it continues to aid and abet my pretension of being a well-read intellectual.


I am extremely fortunate and privileged to have grown up in an atmosphere of love, care, integrity, and excellence, created by my parents Ravindra Gokhale and Mugdha Gokhale who made me who I am today, and my sister Meghana for leading by example.
I am thankful to my grandparents Dr.\ Manohar Joshi and Dr.\ Sulekha Joshi who introduced me to the romance and joy of learning.
Finally, I am thankful to generations of courageous Bharatiyas who liberated my motherland from 1000 long years of terrorism, subjugation, and colonization; without their sacrifices, I would not have had the opportunity of being who I am today.
\end{justify}
}