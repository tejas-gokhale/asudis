\section{SISP Sentence Transformations}
This section describes the generation of semantics-preserving (SP) and semantics-inverting (SI) statements.
\textbf{SISP} transforms are implemented using Spacy~\citep{spacy}.
Dataset statistics and additional visualizations are in the Appendix.

\noindent \textbf{Noun Synonym/Antonym:}
We extract nouns (subjects and objects) with dependency parsing, and find two nearest (synonyms) or farthest (antonyms) neighbors in the GloVe space~\citep{pennington2014glove} using a threshold of $0.55$.

\noindent \textbf{Verb Synonym/Antonym:}
We extract verbs using POS tagging and obtain their synonyms or antonyms.
Verbs are lemmatized and inflected to the correct form using Lemminflect~\citep{lemminflect}.

\noindent \textbf{Comparative Synonym/Antonym:}
Adjectival complements and modifiers are replaced with synonyms (\textit{large $\rightarrow$ big}) or antonyms (\textit{large $\rightarrow$ small}). 

\noindent \textbf{Number Substitution:}
Numerals are replaced by number-words (2 $\rightarrow$ \textit{two}) or vice versa for SP transformations, or by their lower or upper bounds,
(SP: \textit{3 $\rightarrow$ more than two}, SI: \textit{two $\rightarrow$ less than two}).

\noindent \textbf{Pronoun Substitution:}
Human-related nouns (such as \textit{woman, boy, people}) are substituted by pronouns, while pronouns are substituted by generic descriptors (\textit{something, someone, somebody, they}).

\noindent \textbf{Negation:}
We use template-based negation~\citep{gokhale2020vqa} with Subject-Verb Agreement~\citep{wren2000english}.
We add \textit{`did not'} before a past-tense verb, \textit{`do not'}, \textit{`does not'}, or \textit{`not'} before a base-form verb, gerund, or participle, or a \textit{`not'} before an adposition or adjective.

\noindent \textbf{Subject-Object Swap:}
Nominal or clausal subjects and direct or prepositional objects from the sentence are swapped for inverting semantics.

\noindent \textbf{Paraphrasing:}
Input sentences are translated to Russian and then back-translated to English using neural machine translation~\citep{ott2019fairseq}.

\subsection{Data Analysis}
\label{sec:sisp_fidelity_bias}
\paragraph{Quantification of Bias:}
\begin{table}
    \centering
    % \footnotesize
    % \resizebox{\linewidth}{!}{
    \begin{tabular}{@{}l ccc c ccc@{}}
        \toprule
        \multirow{2}{*}{\textbf{Method}} & \multicolumn{3}{c}{NLVR2} & \hphantom & \multicolumn{3}{c}{VIOLIN} \\ 
         \cmidrule{2-4} \cmidrule{6-8}
         & Clean & SP & SI && Clean & SP & SI \\
         \midrule 
        Data-Aug & 51.07 & 50.92 & 40.74 && 61.12 & 62.78 & 62.15 \\
        SW-SDRO  & 51.14 & 50.97 & 40.75 && 62.78 & 58.13 & 64.78 \\
        GW-SDRO  & 51.07 & 50.92 & 40.73 && 62.15 & 52.79 & 74.98 \\
        \bottomrule
    \end{tabular}
    % }
    \caption{Text-only evaluation of biases due to SISP transformations. $50\%$ indicates no bias.}
    \label{tab:eval_textonly_bias}
\end{table}
Since SISP transforms are based on templates, they can potentially introduce spurious linguistic correlations in the dataset.
For example, in NLVR$^2$ and VIOLIN datasets, negations and indefinite pronouns
are infrequent.
To quantify how this could impact models, we mask out the entire image and evaluate models (with VILLA as the backbone for NLVR$^2$ and HERO for VIOLIN).
This acts as a `text-only' evaluation, with accuracies ${\sim}50\%$ implying lesser bias since models do not have access to visual information.
Table~\ref{tab:eval_textonly_bias} shows that SP transforms inflict lesser bias on models than SI transforms.
The effect of bias is dataset-specific; SI makes the prediction of NLVR$^2$ samples harder than random (less than $50\%$ accuracy) but easier for VIOLIN.\\


\noindent\textbf{Transformation Fidelity:}
We employ human subjects to evaluate the quality of SISP-transformed sentences on (1) correctness of labels, (2) grammar, (3) semantics, and (4) visual grounding. 
We report a unified average `transformation fidelity' (details are in Appendix).
Fidelity is higher for SP samples than SI ($90.50\%$ v/s $79.51\%$), which resonates with the complexities of inversion of meaning~\citep{russell1905denoting} and leaves room for improvement in SI transformation.
Although some level of ambiguity exists in SISP transforms, our results show that SDRO models benefit from transformed data.

